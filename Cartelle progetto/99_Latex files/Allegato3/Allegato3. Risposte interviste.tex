\documentclass{article}
    \title{\textbf{Allegato3: Risposte interviste}}
	\date{}

\begin{document}

\maketitle

\paragraph{Intervistato 1}
\begin{itemize}
\item Età: 22
\item Genere: Femmina
\item Domicilio: Capranica (VT), Lazio
\item Metodo intervista: Videochiamata
\end{itemize}
\subparagraph{Risposte}
\begin{enumerate}
\item Ritengo che i beni culturali siano la testimonianza della bellezza del nostro Paese dal punto di vista storico e artistico. Mi capita di visitarli occasionalmente ma la domenica è il giorno perfetto per visitare musei con i propri amici.
\item  Mi piacerebbe visitare molti beni culturali ma dal punto di vista economico questo non è sempre semplice. Si è costretti a scegliere il luogo più agevole anno per anno.
\item Solitamente mi informo tramite il web del luogo, del costo e della distanza. Organizzo con gli amici un piano della giornata per poter visitare più beni culturali possibili nelle vicinanze del luogo dove vogliamo recarci.
\item Cerco sempre di partecipare a tutte le visite possibili ma con la previsione di costi o di impegni di lavoro questo non è sempre possibile.
\item Mi informo sempre in anticipo perché mi piace avere ogni tappa organizzata nel minimo dettaglio.
\item Assolutamente si, mi piacerebbe rimanere aggiornata. Purtroppo mi capita spesso di visitare paesi e città ed essermi persa mostre artistiche le quali non sapevo ci fossero. Secondo me non si pone abbastanza importanza a promuovere le attività riguardanti i beni culturali.
\item Il covid ha influenzato molto il rapporto con i beni culturali. Da quando è arrivata la notizia del virus ho evitato musei o luoghi affollati per evitare contagi, ma questo non ha assopito il desiderio culturale. Quando tutto finirà sarò felice di organizzare una gita con i miei amici nei migliori posti in Italia ed eventualmente nel mondo.
\item Sì, in particolar modo ritengo Trivia una realtà ludica molto educativa sia per adulti che per bambini. Certamente leggere libri o guardare documentari è il modo più bello per apprendere nuove informazioni sui beni culturali, ma non escludo il gioco come un mezzo ricreativo ed utile.
\item Ritengo che l'idea sia molto buona, in modo particolare se questi premi comprendono biglietti/pass per musei o mostre.
\item Assolutamente sì, usufruirei di buoni e di biglietti se questi siano posti in orari comodi.
\end{enumerate}

\paragraph{Intervistato 2}
\begin{itemize}
\item Età: 22
\item Genere: Femmina
\item Domicilio: Farnese (VT), Lazio
\item Metodo intervista: Telefonicamente
\end{itemize}
\subparagraph{Risposte}
\begin{enumerate}
\item Le motivazioni che mi spingono a visitare dei beni culturali sono due: lo faccio per interesse personale e per un interesse nello scoprire un luogo sconosciuto. Le occasioni in cui mi capita di visitarli sono sicuramente durante una vacanza in un’altra città/paese diverso dal mio.
\item Non ho ancora visitato i luoghi che avrei voluto visitare in passato poiché a causa del covid molti luoghi di mio interesse sono stati chiusi.
\item Solitamente mi informo tramite il sito web del bene culturale che mi interessa.
\item Sì, mi è capitato di perdermi una mostra che mi interessava soprattutto perché si trovava lontano da me ed ero impossibilitata a raggiungerla.
\item Sì, solitamente mi informo in anticipo e sarebbe comodo semplificare questa pianificazione.
\item Certo, sapere quale mostra ci sarà nella mia zona sicuramente aumenterebbe la probabilità di andarci.
\item Come ho già risposto nella domanda precedente, il covid ha influenzato il mio rapporto con i beni culturali in modo negativo. L’interesse già lo avevo e di certo non lo ha assopito.
\item Sì, ho già scaricato giochi a quiz sul mio telefono come ad esempio Trivia Crack e penso sia abbastanza utile utilizzarli per imparare cosa nuove.
\item Trovo che sia un’ottima idea.
\item Sì, molto probabilmente la proverei.
\end{enumerate}

\paragraph{Intervistato 3}
\begin{itemize}
\item Età: 22
\item Genere: Maschio
\item Domicilio: Bassano Romano (VT), Lazio
\item Metodo intervista: Telefonicamente
\end{itemize}
\subparagraph{Risposte}
\begin{enumerate}
\item Le scelte che mi spingono a visitare un bene culturale sono l’interesse verso il bene culturale stesso. In base al bene culturale di cui stiamo parlando, mi piace scoprire il lavoro tecnico/artistico delle opere al suo interno o la storia che c’è dietro.
\item Se ho evitato in passato di andare a visitare dei beni culturali è stato soprattutto per una questione di mancanza di tempo libero e di soldi per pagare il biglietto di ingresso.
\item Mi informo quasi esclusivamente tramite il web.
\item Quando sono interessato a qualcosa, mi informo e vedo le mostre che ci saranno in futuro in modo da organizzarmi prima. In generale, non mi sono mai perso nessuna visita.
\item Sì, mi informo sempre in anticipo. Vorrei poterlo fare con semplicità e soprattutto avere tutte le informazioni organizzate in sezioni ed esposte in maniera chiara.
\item Non credo aumenterebbe la frequenza ma potrebbe comunque essere utile e comodo saperlo.
\item No, è più una questione di interesse personale. Il covid non mi ha permesso di visitare beni culturali nell’ultimo anno ma comunque non ha assopito il desiderio di cultura.
\item Sì, credo che sia molto utile ma vedendoli in quell’ottica specifica riterrei necessario che siano riportate delle fonti effettivamente attendibili.
\item Sarebbe un’ottima cosa, porterebbe sicuramente molta più gente a visitare vari beni culturali.
\item Certo a patto che possa scegliere io quali visitare, non andrei mai a vedere qualcosa che non mi interessa.
\end{enumerate}

\paragraph{Intervistato 4}
\begin{itemize}
\item Età: 22
\item Genere: Femmina
\item Domicilio: Oriolo Romano (VT), Lazio
\item Metodo intervista: Di persona
\end{itemize}
\subparagraph{Risposte}
\begin{enumerate}
\item La motivazione che mi spinge a visitare i beni culturali sono l’interesse per l’arte e l’architettura. Li visito soprattutto in estate o comunque quando ho un po’ più di tempo libero, e non in occasioni particolari.
\item Sicuramente avrei voluto visitare la Galleria degli Uffizi a Firenze e Pompei ma purtroppo non ne ho mai avuto la possibilità per una questione di tempo e soldi. Il mio sogno nel cassetto sarebbe comunque quello di andare in Egitto.
\item Solitamente mi informo attraverso il web. Mi è comunque capitato di vedere alcune delle pubblicità in giro per Roma quindi mi sono informata anche da quelle.
\item Sì, è capitato recentemente con la mostra di Caravaggio ai Musei Capitolini e non sono potuta andare soprattutto a causa del Covid e quindi alle difficoltà nel muoversi in questo periodo.
\item Di solito sì, se vado a visitare un bene culturale lo faccio per un mio interesse perciò di solito mi interesso in anticipo. Sarebbe una buona idea il fatto di poterlo fare con maggiore semplicità.
\item Sicuramente. Se le mostre fossero più pubblicizzate sarebbe più facile venirne a conoscenza in tempo, in modo da attirare non solo le persone veramente interessate all'argomento della mostra e che si informano in anticipo, ma anche generiche persone che si potrebbero interessare vedendo magari una semplice pubblicità.
\item Il covid ha bloccato tutto l'ambito turistico, e sinceramente durante il periodo di quarantena mi si è risvegliato l'interesse per scoprire questi posti/mostre/ecc che mi ero persa.
\item Sì, ho scaricato un sacco di giochi a quiz sul telefono e attualmente installati ne ho due che si chiamano Quizland e Cody Cross che non è propriamente un quiz ma è sempre sulla cultura generale. In passato ho giocato molto a Trivia Crack. In generale li reputo ottimi per imparare cose nuove perché ad esempio, quando sbagli una domanda su un argomento che ti interessa vuoi capire qual è la risposta giusta e magari ti viene anche voglia di saperne di più.
\item Dal mio punto di vista sarebbe molto interessante, spingerebbe le persone ad interessarsi di più a questo genere di cose. Se esistesse un app del genere sicuramente la installerei.
\item Se vincessi dei biglietti o dei buoni ne usufruirei senza problemi, potrebbe essere anche una buona scusa per uscire dalla mia confort zone e imparare cose nuove, andando a visitare mostre che magari non mi sarei mai aspettata di visitare perchè non interessata.
\end{enumerate}

\paragraph{Intervistato 5}
\begin{itemize}
\item Età: 30
\item Genere: Maschio
\item Domicilio: Terni (TN), Umbria
\item Metodo intervista: Videochiamata
\end{itemize}
\subparagraph{Risposte}
\begin{enumerate}
\item Le motivazioni sono l'interesse per la cultura e la curiosità di imparare cose nuove specialmente del passato e la passione per la storia. Mi capita di visitare beni culturali solo in occasioni di giornate festive.
\item Non mi viene in mente nessun bene culturale in particolare ma, in ogni caso, se non ci sono potuto andare era soprattutto per questioni di lavoro.
\item Principalmente su internet. Visualizzando da siti o pagine facebook notizie informative riguardanti il sito, gli orari i prezzi d'ingresso ed eventuali recensioni di precedenti visitatori.
\item Raramente ho rinunciato a tali tipo di visite. Avrei voluto fare delle visite guidate a Roma in notturna per le quali mi ero informato ma poi ho rinunciato per questioni lavorative.
\item Sì. Mi informo in anticipo attraverso internet. Trovo già abbastanza facile reperire informazioni dal web su gran parte dei beni culturali più o meno noti.
\item Certamente sì. La pubblicità dei bene culturali aiuta a conoscerne l'importanza ed accresce l'interesse a visitarli.
\item Ha sicuramente risvegliato la voglia di visitare i beni culturali. Quando ci viene proibito di fare qualcosa si ha ancora più voglia di farla e quindi anche di uscire ed andare a visitare qualche museo, villa o altri siti di interesse culturale.
\item Le app contenenti giochi a quiz possono essere utili ad incentivare la curiosità e l'interesse ad informarsi su cose non note o dimenticate perché studiate da giovani. Quindi ritengo utili ed istruttive tali app. Personalmente ho scaricato dei giochi a quiz ma li ho utilizzati per poco tempo, solo per prova ma successivamente li ho disinstallati.
\item Penso che possa essere molto utile per accrescere l'interesse verso i beni culturali e i premi possono essere un ulteriore incentivo a chi è più restio ad avvicinarsi alla tematica.
\item Certamente sì. Ricevere dei buoni per visitare dei beni più o meno noti potrebbe essere l'occasione per non rinviare ulteriormente alcune visite pensate ma poi mai realizzate.
\end{enumerate}

\paragraph{Intervistato 6}
\begin{itemize}
\item Età: 20
\item Genere: Maschio
\item Domicilio: Rieti (RI), Lazio
\item Metodo intervista: Videochiamata
\end{itemize}
\subparagraph{Risposte}
\begin{enumerate}
\item Le visito perché non so per quanto tempo ancora ci saranno, perché comunque rappresentano il passato e il presente di un popolo, della cultura, per ampliare le mie conoscenze.
\item Mi sarebbe piaciuto visitare Machu Picchu, le rovina di una città molto antica, con un panorama bellissimo; mi ha frenato la distanza e il costo del viaggio.
\item Mi informo su internet come mezzo principale.
\item Non mi è mai capitato, sia a causa del fatto che non sono un grande appassionato, sia perché sono sempre state mostre permanenti, non ho mai visto mostre temporanee.
\item Dipende da ciò che vado a vedere, dipende dalla quantità di interesse che ho, penso che in rete le opere non rendano per nulla, dal vivo magari ti fanno impazzire.
\item Nella mia zona non c’è una grande quantità di beni culturali, e soprattutto li valorizziamo poco e male, il FAI sta lavorando per questo, avendo dei beni valorizzati una persona farebbe visite e le farebbe fare magari ad amici che vengono da fuori. Tutti i musei italiani messi insieme guadagnano meno del Louvre.
\item Il Covid ci ha limitato sotto molti punti di vista, sicuramente è rimasta una grande voglia di recuperare il tempo perduto.
\item Si ne ho scaricati molti, ma più che altro per vincere una sfida con gli amici, non tanto per mostrare la propria cultura, basti vedere che molto per vincere baravano (risata), se lo spirito è questo si apprende poco.
\item Andrebbe studiata bene, per evitare imbrogli e trucchi; sarebbe molto interessante.
\item Darei questa possibilità soprattutto a chi non è molto ferrato di cultura, non vedo la necessità per chi è molto bravo di visitare altri beni culturali.
\end{enumerate}

\paragraph{Intervistato 7}
\begin{itemize}
\item Età: 21
\item Genere: Maschio
\item Domicilio: Rieti (RI), Lazio
\item Metodo intervista: Videochiamata
\end{itemize}
\subparagraph{Risposte}
\begin{enumerate}
\item Solitamente me ne interesso più quando sono in vacanza da qualche parte che quando sono nella mia città, quindi tra le motivazioni sicuramente i viaggi turistici con la mia famiglia, mi piace perché sono cose che comunque ho studiato sui libri oppure mi aiutano a capire meglio il posto dove sono.
\item Mi sarebbe piaciuto più che altro vedere cose extra europee, come la Muraglia Cinese o il Taj Mahal, o il Giappone, gli USA; mi hanno frenato più che altro i soldi e la distanza.
\item Su internet solitamente.
\item Forse sì, nei viaggi magari abbiamo tralasciato qualcosa, forse anche perché era chiusa.
\item Dipende, nei viaggi turistici sicuramente un po' si, altrimenti si poteva decidere lì per lì nella città dove si era andati e si decideva.
\item Nella mia provincia i beni culturali sono un po' limitati, comunque si potrebbe interessarmi e potrebbe aumentare la mia frequenza di visita (almeno nella mia regione).
\item Sicuramente ha limitato gli spostamenti e quindi anche la visita ad alcuni posti che mi sarebbe piaciuto visitare, ha ravvivato un po' il mio desiderio di cultura.
\item Sono appassionato di giochi a quiz, ne ho scaricati tantissimi, sia perché sono una persona molto competitiva sia perché serve per imparare chicche che nella vita di tutti i giorni non impareresti mai.
\item Sarebbe una figata, unirebbe la sfida competitiva al poter imparare cose nuove.
\item Probabilmente si, potendo scegliere tra varie opzioni sarebbe carino, magari potendo scegliere tra cosa è più congeniale per noi.
\end{enumerate}

\paragraph{Intervistato 8}
\begin{itemize}
\item Età: 22
\item Genere: Femmina
\item Domicilio: Rieti (RI), Lazio
\item Metodo intervista: Videochiamata
\end{itemize}
\subparagraph{Risposte}
\begin{enumerate}
\item Sono affascinata dal panorama artistico italiano e quando posso mi interesso e visito i luoghi culturali. L’anno scorso quando abitavo a Roma mi sono tolta tanti sfizi e ho visitato spesso chiese, musei, mostre, balletti. Generalmente però lo faccio quando ho occasione di visitare nuovi posti e nuove città.
\item Mi ha frenato sicuramente la disponibilità economica e del tempo, ma tutte le volte che ho viaggiato in nuove città ho visitato tutti i luoghi di interesse artistico principali.
\item Solitamente attraverso social o passaparola.
\item Forse si, o perché prenotata in largo anticipo senza verificare gli impegni futuri di quel giorno o perché dovevo andare con qualcuno che poi non poteva più.
\item Si, a volte anche troppo in anticipo, comunque mi piacerebbe poterlo fare con semplicità.
\item Si certo, se fossi disponibili andrei di più, soprattutto se vicino alla mia zona.
\item Il Covid ha influenzato ogni aspetto della mia vita; spesso in modo negativo, ma non ha spento il mio desiderio di cultura, infatti nonostante tutto sono riuscita ad andare a Roma, in Puglia e in Trentino (sempre rispettando i DPCM).
\item Si, ne ho scaricato qualcuno, sia in italiano che in inglese e le ho ritenute molto utili al fine di apprendere cose nuove.
\item Sarebbe interessante, soprattutto in questo panorama pieno di app di giochi e di nuovi social degradanti.
\item Assolutamente si.
\end{enumerate}

\paragraph{Intervistato 9}
\begin{itemize}
\item Età: 21
\item Genere: Maschio
\item Domicilio: Roma (RM), Lazio
\item Metodo intervista: Videochiamata
\end{itemize}
\subparagraph{Risposte}
\begin{enumerate}
\item Conoscere cose nuove per cultura personale, e poi perché mi piace vedere cose belle, le occasioni sono spesso i viaggi, mi piacerebbe andarci anche qui a Roma, ma dovrei trovare la compagnia adatta.
\item Mi piacerebbe molto visitare i musei Vaticani, mi hanno frenato però il tempo, la compagnia giusta e la voglia; poi ne sto periodo proprio no.
\item Su internet (Google), altrimenti chiedo a qualcuno che ci è andato.
\item Mi è capitato a Londra, volevo andare al British Museum ma ai miei fratelli non andava.
\item Solitamente lo faccio, mi faccio un mini itinerario del posto, e cerco di capire cosa voglio vedere proprio di quel posto, ma solo una infarinata veloce. Riesco a farlo con facilità.
\item Vorrei aumentare un po' la mia frequenza nelle visite artistiche.
\item Poco e nulla, non ci sarei andato molto.
\item Si ne ho scaricate, ma sono scarso e le ho disinstallate, ma le ho usate più che altro per la competizione.
\item Si mi piacerebbe, ci giocherei.
\item Forse si, ma preferirei avere buoni da spendere in luoghi che scelgo io.
\end{enumerate}

\paragraph{Intervistato 10}
\begin{itemize}
\item Età: 26
\item Genere: Maschio
\item Domicilio: Rieti (RI), Lazio
\item Metodo intervista: Videochiamata
\end{itemize}
\subparagraph{Risposte}
\begin{enumerate}
\item Mi capita di visitare beni culturali solo quando viaggio, non ho motivazioni specifiche, ma in certi luoghi alcune tappe sono obbligate (Cfr. Mole a Torino).
\item Forse l’unica che non ho visitato ma che mi sarebbe piaciuto moltissimo era la Sacra Sindone, non sono riuscito perché viene esposta solo in periodi particolari, probabilmente lo farò.
\item Di solito mi informo direttamente sul posto, appena arrivo cerco su internet i luoghi di interesse di quel luogo, e scelgo quelli più congeniali per me, ma anche per come mi sono presentati.
\item Non mi è mai capitato.
\item Si, scrivo su Google, dove trovo informazioni in maniera abbastanza semplice, vorrei semplicemente avere informazioni veloci, ma questo lo si fa già bene. Non mi informo prima di andare solitamente, tranne quando sono andato a Venezia.
\item Forse si, ma i beni culturali della mia città penso di averli visti quasi tutti. Forse però un aggiornamento continuo aumenterebbe la mia frequenza di visita.
\item Assolutamente si, la mancanza di poter viaggiare è la motivazione principale, forse è stato assopito il mio desiderio di cultura.
\item Si, ne ho scaricati, penso che in questi giochi sbagliando o azzeccando le risposte ti rimanga sempre qualcosa, sarebbe interessante farne su argomenti che non si conoscono.
\item Sarebbe carino, ma c’è il rischio che diventi un po' di nicchia (per quanto la nicchia possa essere comunque abbastanza grande), il vincere premi potrebbe aumentare il desiderio di competizione.
\item Penso di si, se faccio un quiz su un argomento che mi piace poi ovviamente lo voglio visitare, gioco perché so cosa posso vincere.
\end{enumerate}

\paragraph{Intervistato 11}
\begin{itemize}
\item Età: 23
\item Genere: Maschio
\item Domicilio: Rieti (RI), Lazio
\item Metodo intervista: Di persona
\end{itemize}
\subparagraph{Risposte}
\begin{enumerate}
\item Mi spingono i miei studi (Liceo Artistico, IED), mi colpisce il tipo di arte che sto andando a vedere, nei musei di arte moderna vorrei vedere cose rivoluzionarie, se vedo un museo Neoclassicista voglio vedere opere interessanti anche esteticamente. Tra i motivi sicuramente la visita di città nuove.
\item Mi ha frenato un po' la pigrizia, un po'  per il cattivo collegamento della mia città.
\item Su internet, in vari siti.
\item Si, un po' perché magari ero impegnato in esami o altro, un po' perché magari erano in altre città e non era facile muoversi.
\item Cerco notizie prima solo se devo vedere luoghi e opere che non conosco, molte cose le so già. Google è già molto semplice.
\item Si, aumenterebbe la mia frequenza.
\item Non mi è cambiato molto da questo punto di vista.
\item Si, è un modo carino per imparare nuove cose.
\item Dipende dai premi che si vincono, e per dove. Se il gioco fosse fatto bene giocherei.
\item Si ne usufruirei, ma potrebbe essere anche una bella idea regalo per qualcuno.
\end{enumerate}

\paragraph{Intervistato 12}
\begin{itemize}
\item Età: 63
\item Genere: Femmina
\item Domicilio: Rieti (RI), Lazio
\item Metodo intervista: Di persona
\end{itemize}
\subparagraph{Risposte}
\begin{enumerate}
\item Il motivo è proprio il desiderio di vedere cose belle, le occasioni sono i momenti liberi, ovvero durante le feste o durante le ferie.
\item Ultimamente il Covid, oppure proprio la mancanza di tempo, perché la voglia c’è sempre.
\item Attraverso siti internet.
\item Sarei voluta andare a vedere Raffaello ma non ci sono potuta andare a causa delle restrizioni anti Covid.
\item Mi informo prima, e quando vado chiedo una guida sul posto.
\item Si, se nelle mie zone specialmente si.
\item Mi ha pesato enormemente, mi ha tolto la possibilità anche di vedere città d’arte a cui ero interessata, però non ha assolutamente affievolito la mia voglia d’arte.
\item Per i giovani potrebbero essere utili, a me non interessano.
\item La pubblicizzerei, sarebbe una buona idea, ma non penso che ne usufruirei. Chiederei a mio marito di farlo per usufruirne anche io.
\item Senza dubbio.
\end{enumerate}

\paragraph{Intervistato 13}
\begin{itemize}
\item Età: 18
\item Genere: Femmina
\item Domicilio: Roma (RM), Lazio
\item Metodo intervista: Di persona
\end{itemize}
\subparagraph{Risposte}
\begin{enumerate}
\item Mi capita di visitare beni culturali per l’interesse di integrare con visite il percorso scolastico, oltre che per le gite scolastiche spesso anche spontaneamente.
\item Le volte che non ho visitato beni culturali è dovuto a problemi logistici o mancata organizzazione.
\item Mi informo su internet e sul sito del bene, cerco principalmente informazioni su orari e sul prezzo del biglietto e su come raggiungere il luogo, non mi informo mai sull'opera preventivamente.
\item Si perché ho saputo tardivamente della chiusura di una mostra.
\item Si come ho già detto precedentemente mi informo sulla logistica, si mi piacerebbe potermi informare con semplicità.
\item Certamente in particolare se si tratta di mostre nella mia zona.
\item Parte delle visite le faccio quando vado in città straniere quindi poiché a causa del covid non ho potuto viaggiare molto ho dovuto rimandare.
\item Sì li apprezzo e sono convinta che sono utili ad imparare nuove cose.
\item Si sopratutto se ho la possibilità di sfidare degli amici.
\item Si in particolare se sono nelle vicinanze.
\end{enumerate}

\paragraph{Intervistato 14}
\begin{itemize}
\item Età: 21
\item Genere: Femmina
\item Domicilio: Roma (RM), Lazio
\item Metodo intervista: Asincrona orale
\end{itemize}
\subparagraph{Risposte}
\begin{enumerate}
\item Visito i beni culturali perché ritengo che sia importante a livello informativo e perché molti hanno un grande rilevanza storica.
\item Non mi piace andare a fare visite da sola, quindi spesso la mancanza di gente con cui andare o l’impossibilità di organizzarsi con amici.
\item Si generalmente mi informo sugli orari e sui prezzi anche se spesso ho difficoltà a trovare queste informazioni.
\item Mi è capitato di perdermi una mostra che mi interessava (di Banksy) perché è saltata a causa del Covid, mentre un'altra volta proprio per le limitazioni del Covid non sono potuta entrare  perché era stata raggiunta la capienza massima.
\item Generalmente mi capita di informarmi grazie ad alcune pagine Facebook a tema o eventi inerenti.
\item Decisamente sì.
\item Sì il Covid ha risvegliato questo desiderio perché ne ho sentito la mancanza.
\item Poche volte, come Trivia Crack perché spinta a giocare da amici, ritengo che siano utili anche se ovviamente le spiegazioni sono brevi e poco dettagliate.
\item Penso che sia interessante un'applicazione del genere.
\item Ne usufruirà ma mi piacerebbe partecipare a quiz sapendo prima cosa si a vincere e che sia di mio interesse.
\end{enumerate}

\paragraph{Intervistato 15}
\begin{itemize}
\item Età: 20
\item Genere: Maschio
\item Domicilio: Roma (RM), Lazio
\item Metodo intervista: Asincrona scritta
\end{itemize}
\subparagraph{Risposte}
\begin{enumerate}
\item Principalmente quando viaggio, oppure raramente quando vengo a conoscenza di mostre o opere di mio interesse a Roma.
\item Probabilmente a causa di imprevisti o mancanza di tempo, principalmente perché non ho avuto modo di andarci e di organizzarmi, valuto anche se il mio interesse vale il costo del biglietto e la fila che trovo sul posto.
\item Cerco su Internet, quando sono fuori vedo quali sono le cose più importanti da vedere (magari dirigendomi anche ad un InfoPoint), vedo come ci si arriva cosa c’è di interessante ed eventualmente il costo del biglietto.
\item Sì, mi è capitato di venire a sapere di una mostra in ritardo quindi quando sono già finite o manca poco tempo alla chiusura, qualche volta mi è capitato di dimenticarmi.
\item Si mi informo sulla storia tramite Wikipedia e su quali sono le cose principali da sapere o da vedere, in realtà mi trovo a cercare su Wikipedia ritengo che ci siano le informazioni sufficienti e scritte abbastanza chiaramente.
\item Forse sì, se non mi dimentico e se ho tempo.
\item Non in modo rilevante, sono rimasto abbastanza indifferente, non credo che in questo periodo avrei fatto delle visite.
\item Sì diverse volte, possono essere utili per imparare oppure per ricordare qualcosa inerente ad un determinato argomento, anche se dopo un pò li ho disinstallati perché ho perso interesse.
\item Penso che sia abbastanza interessante, non so se ci giocherei in ogni caso ritengo che la presenza di premi potrebbe essere un buon incentivo.
\item Se l’opera mi interessa e si trova nella mia città probabilmente si.
\end{enumerate}

\paragraph{Intervistato 16}
\begin{itemize}
\item Età: 20
\item Genere: Maschio
\item Domicilio: Roma (RM), Lazio
\item Metodo intervista: Di persona
\end{itemize}
\subparagraph{Risposte}
\begin{enumerate}
\item Studiando storia dell’arte sono interessato da sempre, quindi se c’è una mostra nella mia città cerco di andarci, oppure mi capita quando vado in vacanza.
\item Dipende, per i beni culturali che mi interessavano al di fuori della mia città perché ancora non ho avuto modo, mentre  per quanto riguarda beni a Roma mi è capitato per motivi di organizzazione o perché non ho trovato nessuno con cui andare, talvolta perché non mi andava.
\item Tramite siti web o volantini, cerco principalmente informazioni su orari e costo del biglietto.
\item Sì per una mia mancata organizzazione.
\item Sì mi informo in anticipo e mi piacerebbe farlo con semplicità,mi informo principalmente quando non conosco l’argomento o il tema  della visita, altrimenti non mi informo ulteriormente, in caso.
\item Sicuramente sì.
\item Sì lo ha influenzato, ne ho risentito anche in ambito universitario (non abbiamo potuto svolgere delle visite previste dal programma), personalmente lo ha risvegliato, perché avevo intenzione di vedere certe cose che non ho ancora visto, cercherò di recuperare quando sarà possibile.
\item Ne ho scaricati diversi, li ritengo utili anche per “ripasso” di argomenti specifici, ci giocavo principalmente con i compagni di classe.
\item Penso che ci sia utile.
\item Sì certo.
\end{enumerate}

\paragraph{Intervistato 17}
\begin{itemize}
\item Età: 21
\item Genere: Femmina
\item Domicilio: Positano (SA), Campania
\item Metodo intervista: Di persona
\end{itemize}
\subparagraph{Risposte}
\begin{enumerate}
\item Mi piace visitare i beni culturali perché rappresentano la nostra storia nel caso dei beni italiani e la storia dell’umanità più in generale. I beni culturali sono tutte le testimonianze, materiali e immateriali aventi valore di civiltà e tra queste sono contemplate l’archeologia, l’architettura, la letteratura, l’arte, la scienza, la bibliografia, l’antropologia. La loro bellezza lascia sempre a chi ha modo di vederle oltre alla loro conoscenza anche una grande ricchezza dell’animo. Mi capita di visitarle durante i viaggi che sono una occasione per conoscere luoghi e storia e vivendo in una zona come la Costiera Amalfitana ho spesso modo di organizzare delle gite per visitare luoghi di grande interesse.
\item Mi è capitato di non poter visitare dei beni culturali a cui ero particolarmente interessata per motivi di tempo in quanto durante un viaggio capita spesso di non riuscire a visitare tutto quello che il luogo offre. Altri motivi sono legati al costo dei biglietti che devono necessariamente comportare una scelta ponderata. La poca disponibilità del tempo e l’aspetto economico incidono sulle scelte data anche la mia condizione di studente.
\item Di solito mi informo tramite i mezzi di informazione o esperienze già fatte da amici e conoscenti. Successivamente approfondisco la mia conoscenza navigando su internet. Se la cosa mi interessa organizzo la visita e cerco di acquistare i biglietti on line per non fare la fila agli ingressi.
\item Si mi è capitato di non poter vedere una mostra a Sorrento perché non sono riuscita a conciliare il trasporto, gli amici e il giorno che potesse andare bene per tutti. Ma spesso capita che non sai dove informarti per vedere se ci sono mostre o eventi nel tuo territorio.
\item Quando pianifico una visita ad un bene culturale preferisco informarmi prima ed ottimizzare i tempi. Amo conoscere prima quello che vado a vedere per poterlo meglio apprezzare e sicuramente farlo in maniera semplificata sarebbe un’ottima soluzione, dato che nella ricerca di siti molto spesso non sono aggiornati, risultano poco chiari e non esplicativi, a volte complessi e macchinosi per quello che ci interessa, perdendo molto tempo nella fase di informazione.
\item Sicuramente qualcosa che mi informasse su mostre ed attività culturali nella mia zona stimolerebbe il mio interesse e probabilmente andrei più spesso ad eventi.
\item Il Covid ha modificato certamente il mio rapporto con i beni culturali perché mi ha impedito fisicamente di poter uscire. Sarebbe stato bello in questo periodo dove si ha molto più tempo a disposizione di poter usufruire di giochi, video, letture e app che ci avessero tenuti legati al mondo dell’arte. Ma penso che dopo questo periodo particolarmente difficile quando tutto finirà molti avranno voglia di riprendere i loro interessi, e nei miei propositi tra le cose da recuperare ci sono anche i beni culturali.
\item Mi è capitato di scaricare giochi e quiz non sull’arte ma su altri argomenti, li ritengo molto utili per imparare cose nuove in quanto anche in momenti di relax e durante i viaggi si può utilizzare il tempo in modo costruttivo divertendosi.
\item Penso che una applicazione a quiz che dia la possibilità di vincere dei premi, incoraggi maggiormente una persona a partecipare soprattutto se i premi consentono di avere delle agevolazioni sugli ingressi delle visite.
\item Sicuramente rappresenta un incentivo perché se sono appassionata di arte antica e vinco un biglietto su una mostra di arte moderna, andandoci potrei appassionarmi anche a quella forma d'arte che non avrei mai pensato di apprezzare. Conoscere ci permette di capire e apprezzare le bellezze che ci circondano e ci consentono di valorizzarle e proteggerle.
\end{enumerate}

\paragraph{Intervistato 18}
\begin{itemize}
\item Età: 20
\item Genere: Femmina
\item Domicilio: Positano (SA), Campania
\item Metodo intervista: Asincrona scritta
\end{itemize}
\subparagraph{Risposte}
\begin{enumerate}
\item Principalmente la curiositá e l'interesse per lo specifico bene culturale. Visito i beni culturali principalmente quando sono in viaggio.
\item Probabilmente ho desistito a causa della troppa distanza e a volte anche dei troppi impegni impegni.
\item Di solito cerco su appositi siti internet che trattano di arte.
\item No non mi é mai capitato.
\item Si mi informo in anticipo per potermi organizzare/gestire. Principalmente organizzo il viaggio per arrivare all'itinerario; mi informo poco su ció che sto andando a vedere, anche se effettivamente mi piacerebbe informarmi.
\item Se sono gratis si.
\item Ha influenzato poiché pur volendo non si può appunto visitare nessun bene culturale.
\item Si, abbastanza utile.
\item Interessante.
\item Si assolutamente.
\end{enumerate}

\paragraph{Intervistato 19}
\begin{itemize}
\item Età: 20
\item Genere: Femmina
\item Domicilio: Sorrento (NA), Campania
\item Metodo intervista: Asincrona scritta
\end{itemize}
\subparagraph{Risposte}
\begin{enumerate}
\item Crescita personale, mi sento fortemente ignorante in materia (ignorante inteso come persona che non sa, che ignora appunto) e quindi mi piace conoscere e crescere in questo senso; mi è capitato quando avevo la visita organizzata da associazioni a cui i miei genitori erano stati invitati a partecipare e quindi mi sono accodata.
\item Pigrizia forse ma non escludo anche una mancanza di organizzazione da parte mia sia per la visita in sé sia per tutto il contorno.
\item Non mi è mai capitato di organizzarla ma se dovessi andare da qualche parte in autonomia senza guida mi informerei su internet, prendendo spunti da più siti.
\item Mai capitato, non sono molto informata sulle visite di durata finita.
\item Mi piacerebbe farlo in semplicità, magari con un percorso di tappe fondamentali.
\item Assolutamente sì, ammetto di non essere informata e diffondere l'informazione mi aiuterebbe innanzitutto ad esserne a conoscenza.
\item In un periodo del genere non andrei mai a mostre o roba simile, cerco di limitarmi il più possibile ma allo stesso tempo ha risvegliato in me una voglia di scoperta e di conoscenza una volta che sarà possibile in maniera tranquilla.
\item Li ho scaricati ma onestamente non so valutare o meno la loro utilità. Magari domande di un argomento che appassiona il singolo può spingere lo stesso a scaricarlo, giocare e apprendere.
\item Assolutamente si, la scaricherei. La questione dei premi può attrarre anche più persone in vista di una ricompensa attirando anche persone a cui l'argomento non appassiona particolarmente.
\item Si, avere i biglietti mi spronerebbe. Mi fa pensare a una cosa organizzata.
\end{enumerate}

\paragraph{Intervistato 20}
\begin{itemize}
\item Età: 21
\item Genere: Femmina
\item Domicilio: Positano (SA), Campania
\item Metodo intervista: Asincrona scritta
\end{itemize}
\subparagraph{Risposte}
\begin{enumerate}
\item Principalmente l'interesse per quel bene culturale, mi capita di visitarli soprattutto durante i viaggi.
\item Se non l'ho fatto è perché non è capitata l'occasione.
\item Mi informo attraverso ricerche su internet.
\item Mi é capitato la maggior parte delle volte durante i viaggi d'istruzione con la scuola per le tempistiche non rispettate.
\item Si generalmente mi informo in anticipo per avere già un'idea di ció che vado a visitare, e si piú è semplice meglio è.
\item Forse, non lo so.
\item Non ha cambiato niente.
\item Se vale quello della patente si. Sono utili secondo me.
\item Sarebbe proprio una buona idea sinceramente.
\item Si se mi interessano quelli in questione altrimenti forse lo darei a qualcun altro più interessato di me.
\end{enumerate}

\paragraph{Intervistato 21}
\begin{itemize}
\item Età: 21
\item Genere: Maschio
\item Domicilio: Positano (SA), Campania
\item Metodo intervista: Di persona
\end{itemize}
\subparagraph{Risposte}
\begin{enumerate}
\item La principale motivazione che mi spinge a visitare i beni culturali è la scoperta del bello, la possibilità che abbiamo attraverso l’arte di arricchire il nostro spirito. Per me, visitare un bene culturale non è una semplice visita ma un momento di riflessione individuale o da condividere con i propri amici. Visito i beni culturali principalmente nei weekend o quando sono libero, da solo o in compagnia di amici. Sicuramente l’iniziativa di rendere gratuito l’accesso a molti beni culturali la prima domenica del mese è un ulteriore stimolo.
\item Essendo uno studente, con limiti come la non totale autonomia non sempre mi è possibile visitare un certo bene culturale, magari perché è troppo lontano rispetto al luogo dove abito poiché avrei difficoltà a raggiungerlo autonomamente. Essendo uno studente e non un lavoratore anche l’aspetto economico della visita secondo me andrebbe considerato, in alcuni casi potrebbe rappresentare un limite.
\item Principalmente per informarmi utilizzo internet, visito alcuni siti ma a volte mi capita di andare insieme ad amici e in alcuni casi mi affido a loro.
\item A volte è capitato di perdere una mostra o una visita che mi interessava, o perché avevo altri impegni (studio, esami ecc…) ma a volte anche per l’impossibilità di recarsi fisicamente nel luogo in questione.
\item A volte capita di informarmi in anticipo, lo faccio con molto piacere se sono particolarmente interessato e incuriosito dal bene culturale che andrò a visitare. Sicuramente una piattaforma digitale, un sito web, un’applicazione che racchiuda informazioni relative ai principali beni culturali del nostro Paese renderebbe tutto più pratico e veloce. Potrebbe essere un’ottima idea!
\item Premettendo che non sono una persona che visita molto frequentemente i beni culturali, un sistema in grado di indicare le mostre nella mia zona potrebbe essere un buon modo per aumentarne la frequenza di visita. Molto spesso, non informandomi in maniera autonoma, non vengo a conoscenza di mostre alle quali sarei anche particolarmente interessato.
\item Con il Covid ho deciso di fermare le visite ai beni culturale per tutela personale e degli altri, anche se giorno dopo giorno cresce il desiderio di tornare a visitare questi posti incantevoli, ricchi di storia, arte e cultura.
\item Si, mi è capitato. Sicuramente sono una modalità alternativa, nuova, rispetto al modo classico con cui impariamo cose nuove. Siamo abituati a formarci e studiare dai libri di testo ed è proprio per questo motivo che questa modalità mi sembra oltre che innovativa anche un approccio diverso, più “informale”, per imparare nuove cose. Molto spesso quello che ci blocca non è imparare una determinata cosa ma il modo in cui la impariamo. A volte siamo restii perché la modalità ci sembra vecchia, noiosa. Ad esempio, ricordo di aver scaricato un’applicazione di nome “Trivia Crack”, un gioco a quiz nel quale avevi la possibilità di sfidare diversi utenti su domande di vario genere. Il modo in cui venivano presentate le domande non era per nulla noioso e incoraggiava l’utente a ragionare. Conoscere nuove cose era piacevole.
\item Certamente è un nuovo modo con cui l’utente può approcciare a questo mondo e interessarsi ai beni culturali. Come ho detto anche prima, un quiz è un modo molto più simpatico, divertente, per scoprire beni culturali che non si conoscono. L’idea dei premi è un plus, incentiverebbe l’utente ad impegnarsi sempre di più e perché no, magari anche ad interessarsi e ad amare questo mondo.
\item Certo, credo che l’idea di mettere in palio, come premi, biglietti o buoni per la visita ai beni culturali sia bellissima. Credo che l’idea di ottenere qualcosa in regalo o tramite una vincita possa attribuire maggiore valore all’oggetto in questione agli occhi di chi lo ha ricevuto. Questo forse può anche stimolare le persone che non sono solite visitare beni culturali.
\end{enumerate}

\paragraph{Intervistato 22}
\begin{itemize}
\item Età: 21
\item Genere: Maschio
\item Domicilio: Positano (SA), Campania
\item Metodo intervista: Asincrona scritta
\end{itemize}
\subparagraph{Risposte}
\begin{enumerate}
\item Le motivazioni sono principalmente la compagnia con cui andare a visitare i beni ma anche la presenza di sconti, iniziative o mostre a tempo limitato nei pressi della mia residenza. Mi capita di visitarli solo in presenza di un gruppo di amici o familiari che propongono la visita.
\item Sostanzialmente per via della distanza del viaggio per arrivare fino al bene, misto al nascere di impegni o problemi imprevedibili che risultano avere una priorità maggiore rispetto alla visita del bene.
\item Per prima cosa visito il sito del presidio dove è ospitato il bene per consultare gli orari di apertura e chiusura e prezzi o, se questo non è esaustivo, quando mi trovo per strada cerco manifesti pubblicitari o opuscoli informativi per raccogliere le suddette informazioni. Dopo essermi informato riguardo la mostra, contatto amici o familiari che possono essere interessati ad accompagnarmi e dopodiché mi informo riguardo ai trasporti per raggiungere il luogo della mostra.
\item Si, e per il momento il motivo principale per cui mi sono perso mostre/visite è stato la grande lontananza e la mia impossibilità a soggiornare in strutture alberghiere per via di costi elevati.
\item Si, prima di ogni visita mi informo sempre in anticipo o, in ogni caso, l'organizzatore di un gruppo di cui faccio parte si informa sempre in anticipo. Sarebbe ottimo poter ottenere sempre informazioni molto semplicemente e soprattutto con largo anticipo proprio per favorire un'organizzazione migliore.
\item Probabilmente sì, conoscendo i programmi di mostre e musei nella mia zona è più facile per me riuscire ad organizzare una visita insieme ad altri membri per via della vicinanza a queste.
\item Si, ha influenzato abbastanza negativamente per via dell'aumentata sedentarietà (e quindi con conseguente mancanza di voglia di muoversi) e per via della vicinanza ad Internet che sto avendo in questo periodo poiché, se sono interessato ad un bene, posso semplicemente cercarlo in rete ed ottenere una grande mole di informazioni e servizi fotografici riguardo il bene senza spostarmi per andare di persona.
\item Si, ho scaricato giochi a quiz sullo smartphone e li reputo abbastanza divertenti e formativi e si, li reputo utili per imparare cose nuove poiché sono potenzialmente a 360 gradi e quindi possono trattare di qualunque argomento e in più sfruttano il divertimento e la competizione per rendere la comprensione molto più leggera dal punto di vista mentale.
\item Penso che sarebbe un buon escamotage per avvicinarle visto che potrebbe passare come un gioco e basta (con in realtà il fine di acculturare le persone) avvicinandole però al mondo dei beni culturali, e il fatto di vincere premi potrebbe essere un incentivo che fa continuare le persone ad usare l'applicazione ed eventualmente anche spargere la voce.
\item Probabilmente sì, dipende dal bene. Ovviamente se si tratta di un bene a cui sono interessato ne usufruirei senza problemi e mi organizzerei di conseguenza, ma se si tratta di un bene non di mio interesse dipende da quanto potrebbe interessarmi e soprattutto dai costi organizzativi per raggiungere il bene
\end{enumerate}

\paragraph{Intervistato 23}
\begin{itemize}
\item Età: 21
\item Genere: Maschio
\item Domicilio: Positano (SA), Campania
\item Metodo intervista: Asincrona orale
\end{itemize}
\subparagraph{Risposte}
\begin{enumerate}
\item Le motivazioni che mi spingono a visitare un bene culturale sono per lo piú quelle di puro interesse riguardo ad uno specifico argomento o uno specifico periodo storico, e le occasioni dove mi capita di visitarle di piú sono durante festivitá natalizie o ferie estive.
\item Il piú delle volte a causa dei costi del viaggio o a causa di una disorganizzazione dell'itinerario di viaggio che mi ha portato ad escludere alcune visite.
\item Mi informo principalmente tramite internet, sui siti dei musei o dei luoghi che vorrei visitare.
\item Si mi é capitato. A causa di impegni di studio o lavorativi, ma anche a volte semplice dimenticanza.
\item Si mi informo il piú possibile prima e vorrei farlo con semplicitá.
\item Ovviamente essere informato in modo rapido aiuterebbe, e soprattutto essere aggiornato tramite i canali giusti.
\item Non l'ha influenzato in nessun modo.
\item Si li reputo utili per imparare, non per imparare da zero cose nuove, ma per consolidare e rafforzare alcune conoscenze giá presenti.
\item Sì sarebbe interessante e promettente.
\item Naturalmente usufruirei dei biglietti ove questo fosse possibile.
\end{enumerate}
 
\end{document}
