\documentclass{article}
    \title{\textbf{Allegato7: Analisi questionario}}
	\date{}
	\usepackage{graphicx}
\begin{document}

\maketitle

\section{Macro-analisi}
Dalle 261 compilazioni del questionario sono stati raffinati i seguenti dati.
\begin{itemize}
\item Visitano annualmente beni culturali:
	\begin{itemize}
	\item 1-2(42.15\%)
	\item 3-5(31.42\%)
	\item piú di 5(19.92\%)
	\item 0(6.51\%)
	\end{itemize}
\item chi li visita lo fa nelle seguenti occasioni:
	\begin{itemize}
	\item viaggio(40.65\%)
	\item weekend(27.48\%)
	\item ferie(16.86\%)
	\item festivita'(15.01\%)
	\end{itemize}
\item e si informa riguardo ai beni culturali tramite:
	\begin{itemize}
	\item passaparola(30.47\%)
	\item social network(25.97\%)
	\item internet(22.9\%)
	\item publicita'(20.65\%)
	\end{itemize}
\item Il problema covid ha inficiato sulla frequenza di visita dei beni culturali:
	\begin{itemize}
	\item Molto(57.09\%)
	\item Abbastanza(29.5\%)
	\item Poco(7.66\%)
	\item Per nulla(5.75\%)
	\end{itemize}
\item Ció che li frena nella visita di beni culturali è:
	\begin{itemize}
	\item tempo(32.65\%)
	\item distanza(23.74\%)
	\item prezzo(17.12\%)
	\item compagnia(12.33\%)
	\item disinteresse(8.68\%)
	\item niente(5.48\%)
	\end{itemize}
\item Scaricano app a tema arte:
	\begin{itemize}
	\item No(69.35\%)
	\item Si(30.65\%)
	\end{itemize}
\item Scaricano giochi a quiz:
	\begin{itemize}
	\item Sì(72.8\%)
	\item No(27.2\%)
	\end{itemize}
\item chi non scarica giochi a quiz é perché:
	\begin{itemize}
	\item disinteresse(82.61\%)
	\item mai avuto occasione(15.94\%)
	\item incapacita'(1.45\%)
	\end{itemize}
\item chi ha risposto si gioca a quiz per:
	\begin{itemize}
	\item Passatempo(34.83\%)
	\item Sfidare gli amici(22.89\%)
	\item Cultura personale(21.39\%)
	\item Spirito di competizione(15.92\%)
	\item Vincere premi(4.98\%)
	\end{itemize}
\item chi ci gioca conosce:
	\begin{itemize}
	\item Trivia Crack(40.05\%)
	\item Quiz duello(37.87\%)
	\item Live Quiz(22.07\%)
	\end{itemize}
\item Parere app quiz tema arte:
	\begin{itemize}
	\item use \& talk(64.14\%)
	\item NOTuse \& talk(23.9\%)
	\item NOTuse \& NOTtalk(11.95\%)
	\end{itemize}
\item Valutano 8.79/10 l’utilitá di vincere dei buoni;
\item Valutano 8.17/10 l’utilitá delle curiositá durante il quiz;
\item Valutano 8.06/10 l’utilitá della bacheca informativa;
\item Valutano 7.92/10 l’utilitá dei temi settimanali;
\item Valutano 7.13/10 l’utilitá dei giochi a quiz;
\item Valutano 6.83/10 l’utilitá della classifica tra amici.
\end{itemize}


\section{Micro-analisi}
La seguente sezione ha il compito di analizzare le risposte al questionario di ciascuno sottogruppo possibile generato dal prodotto cartesiano di genere e etá:
\\\indent
\{Maschio, Femmina, Altro, Generale\} X \{Meno di 18, 18-25, 26-35, 36-45, 46-55, Oltre 55, Generale\}.
\\\indent
Tutte le caratterizzazioni sono state ottenute analizzando esaustivamente il file “results.json” utilizzando lo script “calcoli.py”.
il file “results.json” é il risultato di un attento raffinamento di “table.xlsx” tramite lo script “table.py”. 
“table.xlsx” é il foglio excel ottenuto dal questionario prodotto grazie a google form.
Tutti i suddetti file sono disponibili nella cartella “Materiale microanalisi”.
\\\indent
La microanalisi ha lo scopo di comprendere nel profondo tutti i dati ottenuti dal questionario, in modo tale da poter comprendere a pieno quello che sará il target di riferimento della nostra app, e quali siano i caratteri dei nostri utenti tipo. 
Questo faciliterá le scelte progettuali nella fase di prototipazione, in quanto si avrá a disposizione costantemente anche l’opinione dell’utente, se mai dovessimo averne bisogno e volessimo appellarci a questa.

\end{document}
