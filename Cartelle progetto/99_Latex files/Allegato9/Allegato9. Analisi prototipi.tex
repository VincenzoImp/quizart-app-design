\documentclass{article}
    \title{\textbf{Allegato9: Analisi prototipi}}
   	\date{}
    
\begin{document}

\maketitle

\section{Versione 1}
La seguente versione è stata disegnata sui prototipi cartacei.
\begin{description}
\addtolength{\itemindent}{0.5cm}

\item [Quiz Giornaliero] :
\begin{itemize}
	\item Aspetti negativi riscontrati:
		\begin{itemize}
		\item Occorre implementare la schermata finale visualizzabile al termine della partita;
        \item La barra di progressione si limita ad una sola label che indica qual è la domanda corrente;
        \item Non viene mostrato lo storico dell’esito delle domande;
        \end{itemize}
	\item Aspetti positivi riscontrati:
		\begin{itemize}
        \item Bottoni comprensibili nella schermata iniziale;
        \item Risposte ben leggibili;
        \end{itemize}
	\end{itemize}
\item [Quiz Settimanale]:
	\begin{itemize}
	\item Aspetti negativi riscontrati:
		\begin{itemize}
		\item Colore rosso della schermata iniziale troppo forte;
		\item Difficoltà nel leggere la domanda;
		\item Poca chiarezza della schermata una volta data una risposta;
		\item Schermata finale esteticamente rivedibile;
		\item Notifiche troppo invasive.
	\end{itemize}
	\item Aspetti positivi riscontrati:
	\begin{itemize}
		\item Bottoni comprensibili nella schermata iniziale;
		\item Risposte ben leggibili;
		\item Barra di progressione del quiz.
	\end{itemize}
\end{itemize}

\item [Bacheca] :
\begin{itemize}
	\item Aspetti negativi riscontrati:
		\begin{itemize}
		\item Nella scheda del bene culturale sono presenti troppe scritte che distraggono l’utente e non lo fanno focalizzare sulla comprensione delle informazioni principali;
        \item L’orario è un’implementazione critica in quanto varia in base il giorno della settimana, giorni festivi o eventi speciali, inoltre un bene culturale può essere aperto la mattina, chiudere all’ora di pranzo e poi riaprire nel pomeriggio. Si nota quindi che è difficile informare l’utente in modo dettagliato sotto questo aspetto.
        \end{itemize}
	\item Aspetti positivi riscontrati:
		\begin{itemize}
        \item Pagina della bacheca molto intuitiva grazie alle card con l’anteprima del bene culturale;
        \item Nella scheda del bene culturale sono ben posizionate la foto e la descrizione.
        \end{itemize}
	\end{itemize}
\item [Shop] :
\begin{itemize}
	\item Aspetti negativi riscontrati:
		\begin{itemize}
		\item Difficoltà nella generazione del buono a causa dell’utilizzo di uno slider che non permette di selezionare il valore preciso che si vuole convertire;
		\item Rimangono da definire le informazioni da visualizzare nella schermata dei buoni scaduti.
        \end{itemize}
	\item Aspetti positivi riscontrati:
		\begin{itemize}
        \item Interfaccia semplice e piuttosto intuitiva;
        \end{itemize}
	\end{itemize}
\end{description}

\section{Versione 2}
La seguente versione è stata disegnata sui prototipi cartacei ed è stata poi importata su Marvel.
\begin{description}
\addtolength{\itemindent}{0.5cm}
\item [Generale] :
	\begin{itemize}
	\item Aspetti negativi riscontrati:
		\begin{itemize}
		\item Non viene visualizzata le sezione corrente nella bottom navigation;
		\item Occorre aggiungere la status bar.
	\end{itemize}
	\end{itemize}
	
\item [Quiz Giornaliero] :
\begin{itemize}
	\item Aspetti negativi riscontrati:
		\begin{itemize}
		\item Rimane da definire con certezza cosa inserire all’interno della schermata finale.
        \end{itemize}
	\item Aspetti positivi riscontrati:
		\begin{itemize}
        \item Bottoni comprensibili nella schermata iniziale;
        \item Risposte ben leggibili;
        \item Barra di progressione del quiz;
        \end{itemize}
	\end{itemize}
\item [Quiz Settimanale]:
	\begin{itemize}
	\item Aspetti negativi riscontrati:
		\begin{itemize}
		\item Rimane da definire cosa mostrare nella schermata finale una volta terminato il quiz settimanale.
	\end{itemize}
	\item Aspetti positivi riscontrati:
	\begin{itemize}
		\item Bottoni comprensibili nella schermata iniziale;
		\item Risposte ben leggibili;
		\item Barra di progressione del quiz.
	\end{itemize}
\end{itemize}

\item [Bacheca] :
\begin{itemize}
	\item Aspetti negativi riscontrati:
		\begin{itemize}
		\item Difficoltà nel comprendere il significato di alcune icone come quella del tempo di visita che viene scambiata per il tempo per arrivare al bene culturale;
		\item Nella schermata di attivazione del gps non vengono spiegati i benefici che si guadagnano nell’attivarlo.
        \end{itemize}
	\item Aspetti positivi riscontrati:
		\begin{itemize}
        \item Pagina della bacheca molto intuitiva grazie alle card con l’anteprima del bene culturale;
        \item Nella scheda del bene culturale sono ben posizionate la foto e la descrizione.
        \end{itemize}
	\end{itemize}
\item [Shop] :
\begin{itemize}
	\item Aspetti negativi riscontrati:
		\begin{itemize}
		\item Permangono delle difficoltà nella generazione dei buoni. Gli utenti hanno difficoltà nel comprendere il funzionamento della doppia text box;
		\item Non si riesce a capire il valore della corrispondenza punti/euro;
		\item Vari problemi con il Material Design di Google;
		\item Va modificata la label sulla dialog box della conversione di punti con “annulla” e “converti”;
		\item Occorre modificare la tastiera normale con una numerica;
        \end{itemize}
	\item Aspetti positivi riscontrati:
		\begin{itemize}
        \item Interfaccia semplice e piuttosto intuitiva.
        \end{itemize}
	\end{itemize}
\end{description}

\section{Versione 3}
La seguente versione è quella finale ed è visualizzabile su Marvel.
\begin{description}
\addtolength{\itemindent}{0.5cm}
\item [Generale] :
	\begin{itemize}
	\item Aspetti positivi riscontrati:
		\begin{itemize}
		\item Nella bottom navigation viene visualizzata qual è la sezione corrente;
		\item E' presente la status bar.
	\end{itemize}
	\end{itemize}
	
\item [Quiz Giornaliero] :
\begin{itemize}
	\item Aspetti positivi riscontrati:
		\begin{itemize}
        \item Bottoni comprensibili nella schermata iniziale;
        \item Risposte ben leggibili;
        \item Barra di progressione del quiz.
        \end{itemize}
	\end{itemize}
\item [Quiz Settimanale]:
	\begin{itemize}
	\item Aspetti positivi riscontrati:
	\begin{itemize}
		\item Bottoni comprensibili nella schermata iniziale;
        \item Risposte ben leggibili;
        \item Barra di progressione del quiz.
	\end{itemize}
\end{itemize}

\item [Bacheca] :
\begin{itemize}
	\item Aspetti positivi riscontrati:
		\begin{itemize}
		\item Pagina della bacheca molto intuitiva grazie alle card con l’anteprima del bene culturale;
        \item Nella scheda del bene culturale sono ben posizionate la foto e la descrizione.
        \end{itemize}
	\end{itemize}
\item [Shop] :
\begin{itemize}
	\item Aspetti positivi riscontrati:
		\begin{itemize}
        \item Interfaccia semplice e piuttosto intuitiva;
        \item Maggiore facilità e immediatezza nel convertire un buono.
        \end{itemize}
	\end{itemize}
\end{description} 

\end{document}
