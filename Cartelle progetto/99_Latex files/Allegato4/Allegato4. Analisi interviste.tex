\documentclass{article}
    \title{\textbf{Allegato4: Analisi interviste}}
	\date{}

\begin{document}

\maketitle

\noindent

\section{Macro-analisi}
\begin{itemize}
\item La possibilità di visitare beni culturali principalmente durante ferie/vacanze/viaggi:
\begin{itemize}
\item A causa della distanza dei beni culturali, della mancanza di tempo libero e dal prezzo dei biglietti.
\\\\
\textit{“Mi capita di visitarle durante i viaggi che sono una occasione per conoscere luoghi e storia”} - Intervistato 17
\end{itemize}
\item L’importanza di informarsi su beni culturali attraverso il web:
\begin{itemize}
\item Quasi tutti gli intervistati dicono di usare il web per informarsi.
\\\\
\textit{“Principalmente su internet. Visualizzando da siti o pagine facebook notizie informative riguardanti il sito, gli orari i prezzi d'ingresso ed eventuali recensioni di precedenti visitatori”} - Intervistato 5
\end{itemize}
\item Il pianificare la visita a beni culturali in maniera semplice:
\begin{itemize}
\item Una metà degli intervistati ammette di non avere problemi a pianificare visite a beni culturali, l’altra metà dice di trovare ciò che cerca, ma che preferirebbe farlo in maniera più veloce e semplice.
\\\\
\textit{“Quando pianifico una visita ad un bene culturale preferisco informarmi prima ed ottimizzare i tempi. Amo conoscere prima quello che vado a vedere per poterlo meglio apprezzare e sicuramente farlo in maniera semplificata sarebbe un’ottima soluzione, dato che nella ricerca di siti molto spesso non sono aggiornati, risultano poco chiari e non esplicativi, a volte complessi e macchinosi per quello che ci interessa, perdendo molto tempo nella fase di informazione”} - Intervistato 17
\end{itemize}
\item La possibilità di essere aggiornato in merito a mostre nella propria zona e/o che trattano categorie artistiche di proprio interesse:
\begin{itemize}
\item Questo aumenterebbe la frequenza di visita di beni culturali per i nostri intervistati.
\\\\
\textit{“Assolutamente sì, ammetto di non essere informata e diffondere l'informazione mi aiuterebbe innanzitutto ad esserne a conoscenza”} - Intervistato 19
\end{itemize}
\item L’influenza del Covid-19 nel visitare beni culturali:
\begin{itemize}
\item Gli intervistati che già da prima visitavano beni culturali ammettono di essere un pò intimoriti nel farlo in questo periodo.
\\\\
\textit{“Il covid ha influenzato molto il rapporto con i beni culturali. Da quando è arrivata la notizia del virus ho evitato musei o luoghi affollati per evitare contagi, ma questo non ha assopito il desiderio culturale. Quando tutto finirà sarò felice di organizzare una gita con i miei amici nei migliori posti in Italia ed eventualmente nel mondo”} - Intervistato 1
\end{itemize}
\item Il buon rapporto con i giochi a quiz:
\begin{itemize}
\item Emerge l’importanza nello scaricare giochi a quiz per imparare cose nuove e sfidare i propri amici.
\\\\
\textit{“Sì, ho scaricato un sacco di giochi a quiz sul telefono [...]  In generale li reputo ottimi per imparare cose nuove perché ad esempio, quando sbagli una domanda su un argomento che ti interessa vuoi capire qual è la risposta giusta e magari ti viene anche voglia di saperne di più”} - Intervistato 4
\end{itemize}
\item L’apprezzamento dell’idea di un gioco a quiz a tema ‘beni culturali’:
\begin{itemize}
\item L’idea risulterebbe apprezzata maggiormente se desse la possibilità di vincere buoni e biglietti, ma rimane comunque buona perché porterebbe a imparare cose nuove.
\\\\
\textit{“Certamente è un nuovo modo con cui l’utente può approcciare a questo mondo e interessarsi ai beni culturali. Come ho detto anche prima, un quiz è un modo molto più simpatico, divertente, per scoprire beni culturali che non si conoscono. L’idea dei premi è un plus, incentiverebbe l’utente ad impegnarsi sempre di più e perché no, magari anche ad interessarsi e ad amare questo mondo”} - Intervistato 21
\end{itemize}
\item La possibilità di vincere come premio biglietti e buoni per visitare beni culturali:
\begin{itemize}
\item La metà degli intervistati vorrebbe che fossero personalizzabili per usarli quando e dove vogliono, ma per alcuni invece sarebbe interessante che fossero per luoghi specifici, per imparare qualcosa che altrimenti non imparerebbero, per altri sarebbero una bella idea regalo.
\\\\
\textit{"Se vincessi dei biglietti o dei buoni ne usufruirei senza problemi, potrebbe essere anche una buona scusa per uscire dalla mia confort zone e imparare cose nuove, andando a visitare mostre che magari non mi sarei mai aspettata di visitare perchè non interessata”} - Intervistato 4
\end{itemize}
\end{itemize}


\section{Micro-Analisi}
\begin{itemize}
\item I motivi che spingono a visitare un bene culturale sono:
	\begin{itemize}
	\item 22 persone su 23 per un loro interesse personale;
	\item 1 persona su 23 come passatempo.
	\end{itemize}
\item Le occasioni in cui solitamente si visitano beni culturali sono:
	\begin{itemize}
	\item 6 persone su 23 durante il weekend/festività;
	\item 13 persone su 23 durante una vacanza/gita/visita di una nuova città;
	\item 5 persone su 23 non ritengono di visitarli in occasioni particolari.
	\end{itemize}
\item I fattori che frenano maggiormente nell’andare a visitare un bene culturale sono:
	\begin{itemize}
	\item Mancanza di tempo libero: 9 persone su 23;
	\item I costi: 9 persone su 23;
	\item La distanza: 8 persone su 23;
	\item Impossibilità di visitarli a causa del Covid-19: 2 persone su 23;
	\item Compagnia: 5 persone su 23.
	\end{itemize}
\item In generale si è visto che le motivazioni dietro le quali le persone hanno perso mostre o visite che gli interessavano, sono le stesse che le frenavano ad andare a visitare un bene culturale;
\item Nell’organizzare una visita, i mezzi attraverso i quali le persone si informano sono:
	\begin{itemize}
	\item Il web: 21 persone su 23;
	\item Social network: 2 persone su 23;
	\item Pubblicità: 3 persone su 23;
	\item Passaparola: 5 persone su 23;
	\item Info sul posto: 1 persona su 23.
	\end{itemize}
\item Nella pianificazione di una visita ad un bene culturale:
	\begin{itemize}
	\item Tutti e 23 gli intervistati solitamente si informano in anticipo;
	\item In particolare:
	\begin{itemize}
		\item 10 persone su 23 trovano già tutte le informazioni necessarie per pianificare una visita;
		\item 2 persone su 23 hanno difficoltà a cercare info su un bene culturale e vorrebbero farlo con maggiore semplicità;
		\item 11 persone su 23 trovano già tutte le informazioni necessarie ma gli farebbe comodo farlo con una maggiore semplicità.
		\end{itemize}
	\end{itemize}
\item Per quanto riguarda l’aumento nella frequenza dovuto all’informazione su mostre nelle proprie zone e/o di proprio interesse:
	\begin{itemize}
	\item 20 persone su 23 hanno affermato che questo la aumenterebbe, e di queste:
	\begin{itemize}
		\item 6 dicono che la promozione e la pubblicità aiuterebbero molto nell’aumento della frequenza di visita ai beni culturali;
		\item 5 dicono che se le mostre fossero in zona sarebbe ancora meglio e li spronerebbe di più;
		\item 1 persona andrebbe solo se fossero gratis.
		\end{itemize}
	\end{itemize}
\item Riguardo il rapporto con i beni culturali, nel periodo della pandemia da Covid-19:
	\begin{itemize}
	\item 17 persone su 23 affermano che la pandemia ha molto influenzato il loro rapporto coi beni culturali, e di queste:
	\begin{itemize}
		\item 9 hanno affermato che il proprio desiderio di visitare beni culturali non è stato assopito, inoltre su questi 9:
		\begin{itemize}
		\item 4 dicono che appena la pandemia sarà finita si rifaranno;
		\item 6 hanno affermato che essa ha risvegliato il proprio desiderio di vedere beni culturali;
		\item 2 hanno affermato che questa pandemia ha assopito il loro bisogno d’arte;
		\item 1 lamenta l’impossibilità di viaggiare e di conseguenza di vedere beni culturali;
		\end{itemize}
	\end{itemize}
	\item Le restanti 6 persone su 23 dicono che non sarebbero andate comunque a visitare beni culturali.
	\end{itemize}
\item Nel rapporto degli intervistati con i giochi a quiz è emerso che:
	\begin{itemize}
	\item 22 persone su 23 si dicono interessate, e di queste:
		\begin{itemize}
		\item 15 dicono che li ritengono utili per imparare e/o consolidare le proprie conoscenze;
		\item 5 dicono che l’hanno scaricati principalmente per sfidare i propri amici;
		\item 2 dicono che l’hanno scaricati, ma dopo poco per disinteresse l’hanno disinstallati;
	\end{itemize}
	\item 1 persona su 23 dice di non apprezzare i giochi a quiz.
	\end{itemize}
\item Riguardo all’interesse degli intervistati su un’eventuale app a quiz a tema ‘Beni Culturali’, con vincita di premi, emerge che:
	\begin{itemize}
	\item Tutti e 23 gli intervistati la ritengono una buona idea, e inoltre di questi:
		\begin{itemize}
		\item 8 su 23 si dicono interessati soprattutto per la vincita di premi;
		\item 6 su 23 affermano che porterebbe più persone ad interessarsi ai beni culturali;
		\item 2 persone su 23 si dicono più interessate alla sfida che i giochi a quiz propongono che al tema beni culturali;
		\item 1 persona su 23 non la scaricherebbe, ma la pubblicizzerebbe.
		\end{itemize}
	\end{itemize}
\item Nell'usufruire della vincita di biglietti o buoni per visitare beni culturali, risulta che:
	\begin{itemize}
	\item Tutti li accetterebbero volentieri, però:
		\begin{itemize}
		\item 10 su 23 vorrebbero che i biglietti fossero spendibili nella visita di un bene culturale a propria scelta;
		\item 4 su 23 vorrebbero vincere biglietti che gli propongono visite a beni culturali nuovi e a loro sconosciuti;
		\item 2 persone su 23 ritengono che questi biglietti/buoni potrebbero essere una bellissima idea regalo.
		\end{itemize}
	\end{itemize}
\end{itemize}

\end{document}
